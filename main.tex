\documentclass[12pt]{article}
\usepackage[utf8]{inputenc}
\usepackage[spanish]{babel}
\usepackage{amsmath, amssymb}

\usepackage{geometry}
 \geometry{
 letterpaper,
 total={170mm,200mm},
 left=25mm,
 top=20mm,
 }

\title{Hedging}
\author{German Valle}
\date{November 2022}

\begin{document}

\maketitle

\section{Introducción}

Supongamos que que un portafolio de inversión que al tiempo $t$, tiene una  posición  $\alpha_t$ en acciones denominadas en dólares. Denotemos con $S_t$ al precio de la acción al tiempo $t$. Como no deseamos exposición a dólares es necesario determinar la cobertura óptima que minimice el riesgo de tipo de cambio. 

Sea $FX_t$ al valor del tipo de cambio spot al tiempo $t$, sabemos entonces que el valor de nuestra posición  sin cobertura sería igual a:

\begin{equation}
    V_t = \alpha_tS_tFX_t
    \label{eq:1}
\end{equation}

La intención en este caso es la de cubrir únicamente el riesgo de tipo de cambio mientras conservamos el riesgo de mercado de la acción.  Para ello es necesario entrar en una posición corta en un futuro de FX con con fecha de madurez $T^\star$. Supongamos que esta posición corta es por un total de $\beta_t$ contratos, de modo que al tiempo $t$ el valor de la posición con cobertura es igual a:
\begin{equation}
    \hat{V}_t = \alpha_tS_tFX_t - \beta_t\left(F(t, T^\star) - K\right)
    \label{eq:2}
\end{equation}

Si a lo largo del periodo $[t, T]$ no hubo ningún rebalanceo, al tiempo $T$ el valor de la posición con cobertura es:
\begin{equation}
    \hat{V}_T = \alpha_tS_TFX_T - \beta_t\left(F(T, T^\star) - K\right)
    \label{eq:3}    
\end{equation}

El PnL de esta posición a lo largo del intervalo $[0,T]$ sería:
\begin{equation}
    \begin{split}
        \hat{V}_T - \hat{V}_t  & = \alpha_t\left(S_TFX_T - S_tFX_t\right) - \beta_t\left(F(T, T^\star) - F(t, T^\star)\right)\\
        & = \left(\alpha_tS_TFX_T - \beta_tF(T, T^\star)\right) - \left(\alpha_tS_tFX_t  - \beta_tF(t, T^\star)\right)
    \end{split}
        \label{eq:4}
\end{equation}
Una selección común para realizar esta cobertura es hacer:
\begin{equation}
    \beta_t = \alpha_tS_t
    \label{eq:5}
\end{equation}
La idea de esta selección es la de cubrir al 100\% el riesgo de FX. Sin embargo esto no siempre es óptimo, tal como veremos mas tarde.

La ecuación (\ref{eq:4}) se reescribe como:
\begin{equation}
\begin{split}
            \hat{V}_T - \hat{V}_t  & = \alpha_t\left(S_TFX_T - S_tFX_t\right) - \alpha_tS_t\left(F(T, T^\star) - F(t, T^\star)\right) \\
            & = \alpha_t\left(S_TFX_T - S_tF(T, T^\star)\right) - \alpha_tS_t\left(FX_t - F(t, T^\star)\right)
\end{split}
        \label{eq:6}
\end{equation}

 Notemos en (\ref{eq:6}) que  la aleatoriedad del PnL se debe al término:
 \begin{equation}
    \alpha_t\left(S_TFX_T - S_tF(T, T^\star)\right)    
    \label{eq:7.1}
 \end{equation}
 mientras que el término:
 \begin{equation}
     \alpha_tS_t\left(FX_t - F(t, T^\star)\right)
     \label{eq:7.2}
 \end{equation}
 es completamente determinístico.
 
El problema en (\ref{eq:7.1})  es que en realidad no existe modo de elimiar el efecto del tipo de cambio en el riesgo de la posición. Aun cuando  $T = T^\star$ y por tanto $F(T, T^\star) = FX_T$ el factor estocástico en el PnL toma la forma:
 \begin{equation}
     \alpha_t\left(S_TFX_T - S_tF(T, T^\star)\right) = \alpha_tFX_T\left(S_T - S_t\right) 
     \label{eq:7.3}
 \end{equation}
 Este factor desaparece solo cuando $S_t = S_T$, lo cual es completamente improbable. Consideremos además que que consideramos solo el caso donde $T = T^\star$, pero lo cierto es que dada la naturaleza del trade, no existe una fecha de madurez. Esto da lugar a un riesgo de base permanente que ni siquiera una recalibración frecuente de la cobertura corrige satisfactoriamente.

En este contexto, ¿como determinamos el hedge óptimo? Lo primero es definir el cociente:

\begin{equation}
    \phi = \frac{\alpha_t}{\beta_t}
    \label{eq:8}
\end{equation}
y para simplificar usaremos la notación:
\begin{equation}
    \Tilde{S}_t = S_tFX_t
    \label{eq:9}
\end{equation}
De este modo, reescribimos (\ref{eq:4}) como
\begin{equation}
    \begin{split}
            \hat{V}_T - \hat{V}_t  & = \beta_t\left(\tilde{S}_T - \tilde{S}_t\right) - \beta_t\phi\left(F(T, T^\star) - F(t, T^\star)\right) \\
            & = \beta_t\left(\tilde{S}_T - \tilde{S}_t - \phi\left(F(T, T^\star) - F(t, T^\star)\right)\right)
    \end{split}
    \label{eq:10}
\end{equation}
denotamos con:
\begin{equation}
    \Delta \tilde{S} = \tilde{S}_T - \tilde{S}_T \quad \text{ y } \quad \Delta F = F(T, T^\star) - F(t, T^\star)
    \label{eq:11}
\end{equation}
De modo que:
\begin{equation}
    \hat{V}_T - \hat{V}_t = \beta_t\left(\Delta\tilde{S} - \phi\Delta F\right)
    \label{eq:12}
\end{equation}
Si usamos a la varianza como medida de riesgo, el objetivo es seleccionar el valor de $\phi$ que minimice la varianza de (\ref{eq:12})

Tenemos entonces que:
\begin{equation}
	\begin{split}
		Var\left(\hat{V}_T - \hat{V}_t\right) & = \beta^2_tVar\left(\Delta\tilde{S} - \phi\Delta F\right) \\ 
		                                      & = \beta^2_t \left(Var(\Delta\tilde{S}) - \phi^2Var(\Delta F) +2 \phi Cov(\Delta \tilde{S},\Delta F)\right) \\ 
	\end{split}
	\label{eq:13}
\end{equation}
Para minimizar la varianza, calculamos las condiciones de primer orden: 
\begin{equation}
	-\phi Var(\Delta F) +Cov(\Delta \tilde{S},\Delta F) = 0
	\label{eq:14}
\end{equation}

De donde se sigue que:

\begin{equation}
	\phi = \frac{Cov(\Delta \tilde{S},\Delta F)}{Var(\Delta F)} 
	\label{eq:15}
\end{equation}
o bien

\begin{equation}
	\phi = \rho\frac{\sigma(\Delta \tilde{S})\sigma(\Delta F)}{\sigma^2(\Delta F)} 
	\label{eq:15}
\end{equation}



\end{document}
